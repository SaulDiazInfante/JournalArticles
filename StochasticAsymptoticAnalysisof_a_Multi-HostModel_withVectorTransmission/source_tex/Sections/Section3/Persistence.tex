	In the above section, we give conditions to assure disease extinction. 
Now, we study other important characteristic in epidemiology ---~the
disease persistence. We know that if $\mathcal{R}_{0}>1$, 
then system~\eqref{eq.1} has an unique endemic equilibrium. 
Four stochastic extension \eqref{eq.2}, we give conditions to guarantee 
infected population persistence in order to establish disease presence. 
To this end, first we enunciate the following.
\begin{definition}[{\citet[pg.~936]{Liu2010}}] \label{def.pers}
	Let us denote by $p(t)$ the  population size at time $t$. 
	If
	$
		\liminf\limits_{t\rightarrow\infty}
		\frac{1}{t}
		\int\limits_{0}^{t}p(s)ds >0
	$
	 a.s., then $p(t)$ is said to be stochastic strongly persistent in the mean.
\end{definition}
%
\begin{lemma}[{\citet[Lemma 5.1]{Ji2014}}]\label{lem:log_Gronwall}
		Let $p\in \mathcal{C}([0,+\infty)\times\Omega,(0,+\infty))$. If there exist 
	positive constants 
	$
		\lambda_{0}
	$, 
	$
		\lambda
	$, such that 
	$$
		\ln(p(t))\geq 
			\lambda t - \lambda_{0}
			\int\limits_{0}^{t} 
				p(s)
			ds 
		+ 
		F(t),\text{ a.s.}
	$$
	for all $t\geq 0$, where 
	$
		F\in \mathcal{C}([0,+\infty)\times\Omega,\mathbb{R})
	$ 
	and 
	$
		\lim
			\limits_{t\to \infty}\frac{F(t)}{t}=0
	$ 
	a.s.,
	 then
	\begin{align*}\label{lem.marting}
		\liminf
		\limits_{t\rightarrow\infty}
			\frac{1}{t}
			\int\limits_{0}^{t}
				p(s)
			ds 
			&\geq 
				\frac{\lambda}{\lambda_{0}}\text{ a.s.}
	\end{align*}
\end{lemma}
To simplify notation, we write
\begin{align*}
	c_{1}= 
		\min
		\left\{
			\alpha_{h}H
			+\alpha_{a}A,
			\frac{\alpha_{v_{h}}K}{2},
			\frac{\alpha_{v_{a}}K}{2}
		\right\},
		&&
		c_{2}= 
			\max
			\left\{
				\mu_{h},
				\mu_{a}, 
				\Lambda_{v}
			\right\},
		&&
		c_{3}= 
			\max
				\left\{ 
					c_{31}, c_{32}, c_{33},
					c_{34}, c_{35} 
				\right\},
\end{align*}
where,
\begin{align*}
	&c_{31} = 
		\sup
			\limits_{(I_{h},I_{a},S_{v},I_{v})\in\mathbf{D}}
			\left\{
				F_{1}^{2}\theta_{h}^{2}H^{2}+F_{2}^{2}\theta_{a}^{2}A^{2}
			\right\}, 
	&&c_{32} = \sup\limits_{(I_{h},I_{a},S_{v},I_{v})\in\mathbf{D}}
		\left\{
			F_{1}^{2}\theta_{v_{h}}^{2}K^{2}
		\right\},\\
	&c_{33} = 
		\sup
			\limits_{(I_{h},I_{a},S_{v},I_{v})\in\mathbf{D}}
			\left\{
				F_{2}^{2}\theta_{v_{a}}^{2}K^{s2}
			\right\},
	&&c_{34} = 
		\sup
			\limits_{(I_{h},I_{a},S_{v},I_{v})\in\mathbf{D}}
			\left\{F_{1}^{2}\theta_{h}H\theta_{v_{h}}K\right\},\\
	&c_{35} = 
		\sup\limits_{(I_{h},I_{a},S_{v},I_{v})\in\mathbf{D}}
		\left\{
			F_{2}^{2}\theta_{a}A\theta_{v_{a}}K
		\right\}.
\end{align*}
%
\begin{theorem}\label{theo:persist}
	Let 
	$
	\left(
	I_{h}(t_0),, I_{a}(t_0), S_{v}(t_0), I_{v}(t_0)
	\right)
	\in \mathbf{D}$. 
	If 
	$2c_{1}-2c_{2}-c_{3}>0$, then the infected population of
	SDE~\eqref{eq.2}
	$
			I_{h}(t) +I_{a}(t) + I_{v}(t)
	$ is stochastic strong persistent in the mean.
\end{theorem}
\begin{proof}
		The main idea of the proof is to apply 
		\Cref{lem:log_Gronwall}. To this 
	end, we use our constants $c_1$, $c_2$, $c_3$ in 
	order to argue the required hypothesis.
	By It\^o's  formula
	\begin{align*}
		d(\ln(I_{h}+I_{a}+I_{v})) &= 
		\left[
			\left(
				\frac{
					\alpha_{h}I_{v}(H-I_{h})
					-\mu_{h}
				}{I_{h}+I_{a}+I_{v}}
			\right)
			+
			\left(
				\frac{
					\alpha_{a} I_{v} (A-I_{a})
					-\mu_{a}I_{a}
				}
				{
					I_{h} + I_{a} + I_{v}
				}
			\right)
		\right.
			\\ %
			&+
		\left.
			\left( 
				\frac{
					(
						\alpha_{v_{h}} I_{h}
						+ \alpha_{v_{a}} I_{a}
					)
					S_{v}
				}
				{
					I_{h}
					+I_{a}
					+I_{v}
				}
			\right)
			-
			\left(
				\frac{
					(
						\mu_{v} 
						+r_{k} T_{v}
					) I_{v}
				}
				{
					I_{h} 
					+ I_{a}
					+I_{v}
				} 
			\right)
		\right] 
		dt
		\\ 
		&-
		\frac{1}{2}
		\left[
			\left(
				\frac{
					F_{1}
					(
						\theta_{h} I_{v} (H-I_{h})
						+\theta_{v_{h}} I_{h} S_{v}
					)
				}
				{
					I_{h} + I_{a} + I_{v}
				}
			\right)^{2}
			+
			\left(
				\frac{
					F_{2}
					(
						\theta_{a} I_{v} 
						(A - I_{a})
						+ \theta_{v_{a}} I_{a} S_{v}
					)
				}
				{
					I_{h} + I_{a} + I_{v}
				}
			\right)^{2}
		\right]
		dt
		\\ %
		&+ 
		\left( 
			\frac{
				F_{1}
				(
					\theta_{h} I_{v} (H-I_{h})
					+\theta_{v_{h}} I_{h} S_{v}
				)
			}
			{
				I_{h} + I_{a} + I_{v}
			}
		\right)
		dW_{1}
		+
		\left( 
			\frac{
				F_{2}
				(
					\theta_{a} I_{v} (A-I_{a})
					+\theta_{v_{a}} I_{a} S_{v}
				)
			}
			{
				I_{h} + I_{a} + I_{v}}
		\right)
		dW_{2}  ~.
	\end{align*}
%
	Next, using $-(\mu_{v} +r_{k}T_{v})\geq -\Lambda_{v}$, we get
	\begin{align*}
		d(
			\ln(I_{h} + I_{a} + I_{v})
		) 
		&\geq 
			\left[ 
				\left(
					\frac{
						\alpha_{h} I_{v} (H-I_{h})
						-\mu_{h}} {I_{h} + I_{a} + I_{v}}
				\right)
				+
				\left(
					\frac{
						\alpha_{a} I_{v} (A-I_{a})
						-\mu_{a} I_{a}
					}
					{
						I_{h} + I_{a} + I_{v}
					}
				\right)
			\right.
				\\ %
				&-
			\left.
				\left(
					\frac{
						\Lambda_{v} I_{v}
					}
					{
						I_{h} + I_{a} + I_{v}
					}
				\right)
				+
				\left(
					\frac{
						(
							\alpha_{v_{h}} I_{h}
							+ \alpha_{v_{a}} I_{a}
						)
						\left(
							T_{v} - I_{v}
						\right)
					}
					{
						I_{h} + I_{a} + I_{v}
					}
				\right)
			\right]
			dt 
			\\ &
			- 
			\frac{1}{2}
			\left[
				\left(
					\frac{
						F_{1}
						(
							\theta_{h} I_{v} (H-I_{h})
							+\theta_{v_{h}} I_{h} S_{v}
						)
					}
					{
						I_{h} + I_{a} + I_{v}
					} 
				\right)^{2}
				+
				\left(
					\frac{
						F_{2}
						(
							\theta_{a} I_{v} (A-I_{a})
							+\theta_{v_{a}} I_{a} S_{v}
						)
					}
					{
						I_{h} + I_{a} + I_{v}
					}
				\right)^{2}
			\right]
			dt
			\\
			&+
			\left(
				\frac{
					F_{1}
					(
						\theta_{h} I_{v} (H-I_{h})
						+\theta_{v_{h}} I_{h} S_{v}
					)
				}
				{
					I_{h} + I_{a} + I_{v}
				}
			\right)
			dW_{1}
			+
			\left( 
				\frac{
					F_{2}
					(
						\theta_{a} I_{v} (A-I_{a})
						+\theta_{v_{a}} I_{a} S_{v})
				}
				{
					I_{h} + I_{a} + I_{v}
				}
			\right)
			dW_{2} ~.
	\end{align*}
%
	Applying definition of $c_{2}$ and $c_{3}$, we have
	\begin{align*}
		d(\ln(I_{h}+I_{a}+I_{v})) 
			&\geq 
				\left(
					\frac{
						\left(
							\alpha_{h}H + \alpha_{a} A 
						\right) I_{v}
						+
						(
							\alpha_{v_{h}} I_{h}
							+ \alpha_{v_{a}} I_{a}
						) 
						T_{v}
					}
					{
						I_{h} + I_{a} + I_{v}
					}
				\right)
				dt
				-
				\left(
					\frac{
						\left(
							\alpha_{h} + \alpha_{v_{h}}
						\right)
						I_{v} I_{h}
					}
					{
						I_{h} + I_{a} + I_{v}
					}
				\right)
				dt
				\\
				&-
				\left(
					\frac{
						\left(
							\alpha_{a} + \alpha_{v_{a}}
						\right)
						I_{v} I_{a}
					}
					{
						I_{h} + I_{a} + I_{v}
					}
				\right)
				dt
				- c_{2} dt
				-
				\frac{c_{3}}{2}
				\left(
					\frac{
						I_{v}^{2} + I_{h}^{2} 
						+ I_{a}^{2} + 2I_{v}I_{h} + 2I_{v}I_{a}
					}
					{
						(
							I_{h} + I_{a} + I_{v}
						)^{2}}
					\right)
					dt\\
				&+
				\left(
					\frac{
						F_{1}
						(
							\theta_{h} I_{v} (H-I_{h})
							+ \theta_{v_{h}} I_{h} S_{v}
						)
					}
					{
						I_{h} + I_{a} + I_{v}
					}
				\right
				)
				dW_{1}
				+
				\left(
					\frac{
						F_{2}
						(
							\theta_{a} I_{v} (A-I_{a})
							+ \theta_{v_{a}} I_{a} S_{v})
					}
					{
						I_{h} + I_{a} + I_{v}
					}
				\right)
				dW_{2} ~.
	\end{align*}
	We now integrate on both sides to deduce that,
	\begin{align*}
		\ln(I_{h} + I_{a} + I_{v})
		&\geq
			\ln(
				I_{h}(0) + I_{a}(0) + I_{v}(0)
			)
			-
			\left(
				c_{2} 
				+
				\frac{c_{3}}{2}
			\right) t 
			+
			\int
				\limits_{0}^{t}
					\left(
						\frac{
							\left(
								\alpha_{h}H + \alpha_{a}A
							\right)I_{v}
						}
						{
							I_{h} + I_{a} + I_{v}
						}
					\right
				)ds
			\\
			&+ 
			\int\limits_{0}^{t}
				\left(
					\frac{
						(
							\alpha_{v_{h}} I_{h}
							+ \alpha_{v_{a}} I_{a})T_{v}
					}
					{
						I_{h} + I_{a} + I_{v}}
				\right)
				ds
				-
				\left( 
					\frac{
						\alpha_{h} + \alpha_{v_{h}} 
						+ \alpha_{a} + \alpha_{v_{a}}
					}
					{4} 
				\right)
				\int\limits_{0}^{t}
					(
						I_{h}+I_{a}+I_{v}
					)
				ds
			\\
			&+
			\int
				\limits_{0}^{t}
				\left(
					\frac{
						F_{1}
						(
							\theta_{h} I_{v} (H-I_{h})
							+ \theta_{v_{h}} I_{h}S_{v})
					}
					{
						I_{h} + I_{a} + I_{v}
					}
				\right)
				dW_{1}(s)%\\
				+
				\int
				\limits_{0}^{t}
					\left(
						\frac{
							F_{2}
								(
									\theta_{a} I_{v} (A-I_{a})
									+ \theta_{v_{a}} I_{a} S_{v})
						}
						{
							I_{h} + I_{a} + I_{v}}
					\right)
					dW_{2}(s) ~.
	\end{align*}
	Since function $T_{v}(\cdot)$ is increasing  for all initial condition 
	$T_{v}(0)\in (0,K]$, there exists $t_{K}\geq 0$, such that 
	$$
		T_{v}(s) \geq \frac{K}{2}, \quad \text{ for all } s\geq t_{K} ~.
	$$
	Consequently, if $t\geq t_K$, then
	\begin{align}%\label{eqn:to_bound_mith_M}
		\ln(
			I_{h} + I_{a} + I_{v}
		)
			&\geq 
				\ln(
					I_{h}(0)+I_{a}(0)+I_{v}(0)
				)
				-
				\left(
					c_{2}
					+
					\frac{c_{3}}{2}
				\right)t 
				+
				\int
				\limits_{t_{K}}^{t}
					\left(
						\frac{
							\left(\alpha_{h}H 
							+\alpha_{a} A
							\right)I_{v}
						}
						{
							I_{h} + I_{a} + I_{v}
						}
					\right)
					ds \notag
					\\
				&+
				\int
					\limits_{t_{K}}^{t}
						\left(
							\frac{
								(
									\alpha_{v_{h}} I_{h}
									+ \alpha_{v_{a}} I_{a}
								)
								\frac{K}{2}
							}
							{
								I_{h}+I_{a}+I_{v}
							}
						\right)
				ds
				-
				\left(
					\frac{
						\alpha_{h} + \alpha_{v_{h}}
						+ \alpha_{a} + \alpha_{v_{a}}
					}{4}
				\right)
				\int \limits_{0}^{t}
				(
					I_{h} + I_{a} + I_{v}
				)
				ds 
				\\
				&+
				\int\limits_{0}^{t}
					\left(
						\frac{
							F_{1}
							(
								\theta_{h} I_{v} (H-I_{h})
								+ \theta_{v_{h}} I_{h}S_{v})
						}
						{
							I_{h} + I_{a} + I_{v}
						}
					\right)
					dW_{1}(s)
				+
				\int
				\limits_{0}^{t}
				\left(
					\frac{
						F_{2}
						(
							\theta_{a} 
							I_{v} (A-I_{a})
							+ 
							\theta_{v_{a}} I_{a}S_{v}
						)
					}
					{
						I_{h} + I_{a} + I_{v}
					}
				\right)dW_{2}(s) ~. \notag
	\end{align}
	For abbreviation, we write
	\begin{align*}
		M_0^{t_K} &:=
		\ln(I_{h}(0) + I_{a}(0) + I_{v}(0)) - c_{1}t_{K},
		\\ 
		M_1(t) &:=
			\int\limits_{0}^{t}
			\left(
				\frac{F_{1}
				(
					\theta_{h} I_{v}
					(H-I_{h})
					+\theta_{v_{h}}I_{h}S_{v})
				}
				{
					I_{h} + I_{a} + I_{v}
				}
			\right)
			dW_{1}(s)
			+
			\int\limits_{0}^{t}
			\left(
				\frac{
					F_{2}
					(
						\theta_{a}I_{v}(A-I_{a})
						+\theta_{v_{a}}I_{a}S_{v}
					)
				}
				{
					I_{h} + I_{a} + I_{v}}
			\right)dW_{2}(s) ~.
	\end{align*}
	Since $t_K$ is fixed and finite, we see that
	$$
		\lim\limits_{t \to \infty}
			\frac{M_0^{t_K}}{t} = 0
			~.
	$$
	Furthermore, the strong law of large numbers for martingales 
	\citep[see e.g.,][Thm. 3.4]{Mao2007} implies
	\begin{align*}
		\lim
			\limits_{t \to \infty}\frac{M_1(t)}{t}=0  \quad \text{a.s.}
	\end{align*}
	%
	Note that by hypothesis $ 4c_1 - 4 c_2 - 2 c_3 >0$.
	Therefore, applying \Cref{lem:log_Gronwall},
	with 
	\begin{align*}
		p(t)= 
			I_{h}(t)
			+I_{a}(t)
			+I_{v}(t),
		&&F(t)= 
			M_0^{t_K} + M_1(t), 
		&&\lambda_0 =
				\alpha_{h}
				+\alpha_{a} 
				+\alpha_{v_{h}}
				+\alpha_{v_{a}},
		&&\lambda = 
			4c_1 - 4 c_2 - 2 c_3,
	%
	\end{align*}
	we show stochastic persistence in the mean for our infected populations.
\end{proof}