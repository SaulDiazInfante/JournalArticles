	In an epidemiology context, one of the goals is to study  sceneries where a 
disease extinguishes or persists. So in the following sections we stay 
results in this direction. Let the threshold parameter
\begin{align*}
	\mathcal{R} &:= 
	\frac{
		\alpha_{h} \alpha_{v_{h}}
	}{
		\mu_{h} \mu_{v}
	} H K
	+
	\frac{
		\alpha_{a} \alpha_{v_{a}} 
	}{
		\mu_{a} \mu_{v}
	} A K.
\end{align*}
In the next theorem, we give suitable conditions related with $\mathcal{R}$ 
value to assure disease extinction. That is, under definition of 
stochastic asymptotic stability in the large (definition \autoref{dfn:app3}) 
and conditions of the SDE \eqref{eq.2} parameters
we establish this kind of stability for the disease-free equilibrium point.

\begin{theorem}\label{theo:GSAS}
		If $\mathcal{R}<1$, then the disease-free equilibrium point, 
	$x^{*}=(0,0,K,0)$, of SDE~\eqref{eq.2} is globally stochastically 
	asymptotically stable in the large.
\end{theorem}
\begin{proof}
	The main idea of the proof is to propose a Lyapunov function $(V)$ and verify 
	the hypotheses of  \Cref{theo.ap.3}. Let 
	\begin{align*}
		V(I_{h},I_{a},S_{v},I_{v}) 
			&:=
			\frac{1}{2}\left( S_{v}+I_{v} - K \right)^{2}
			+p_{1}I_{h} + p_{2}I_{a} + I_{v},		\\
		p_{1} 
			&=
			\frac{\alpha_{v_{h}}}{\mu_{h}} K,		\quad
		p_{2}
			=
			\frac{\alpha_{v_{a}}}{\mu_{a}} K.
	\end{align*}
	Note that function $V(I_{h},I_{a},S_{v},I_{v})$ is non-negative 
	definite on $\mathbf{D}$. 
	In order to satisfies \Cref{theo.ap.3}, we show that function 
	$V(I_{h},I_{a},S_{v},I_{v})$ is radially unbounded. 
	Define
	\begin{align*}
		q &= \min \left\lbrace \frac{1}{2},p_{1},p_{2} \right\rbrace,
	\end{align*}
	and arbitrarily fix $M>0$. Taking 
	$
		N > \max\left\lbrace \frac{2M+K^{2}}{q}, 
		\sqrt{\frac{8M+4K^{2}}{q}} \right\rbrace >0
	$
	 such that 
	$
		N < \norm{(I_{h},I_{a},S_{v},I_{v})}
	$ and 
	non negative
	$
		I_{h},I_{a},S_{v},I_{v}, 
	$
	we have that
	\begin{align*}
		\frac{1}{2}
		\left(
			S_{v}+I_{v}-K 
		\right)^{2} 
		+ p_{1}I_{h} 
		+ p_{2}I_{a} 
		+ I_{v} > M.
	\end{align*}
	
	Hence, letting $M \to \infty$, we deduce that
	$
		V(I_{h},I_{a},S_{v},I_{v})
	$
	is radially unbounded in the sense of \Cref{dfn:radial_unbonded}.
	Now, applying the infinitesimal generator $\mathcal{L}$ (defined on 
	\ref{appe.1}) to function $V(I_{h},I_{a},S_{v},I_{v})$, we obtain
	\begin{align*}
		\mathcal{L}V(I_{h},I_{a},S_{v},I_{v}) 
			&= p_{1} 
			\left[ 
				\alpha_{h}I_{v}
					\left(
						H-I_{h}
					\right)
					-\mu_{h}I_{h} 
			\right] 
			+ p_{2} 
			\left[
				\alpha_{a}I_{v}
				\left(
					A-I_{a}
				\right)
				-\mu_{a}I_{a} 
			\right]\\
			&+ 
			\left(
				S_{v}+I_{v}-K 
			\right)
			\left[
				\Lambda_{v} T_{v}
				-
				\left(
					\alpha_{v_{h}}I_{h}
					+\alpha_{v_{a}}I_{a}
				\right)S_{v}
				-
				\left(
					\mu_{v} 
					+r_{k}T_{v}
				\right)S_{v}
			\right]
			\\
			&+
			\left(
				S_{v}
				+I_{v}-K
			\right)
			\left[
				\left(
					\alpha_{v_{h}}I_{h}
					+\alpha_{v_{a}}I_{a}
				\right)
				S_{v}
				-
				\left(
					\mu_{v}
					+r_{k}T_{v}
				\right)
				I_{v}
			\right]
			\\
			&+ 
			\left[
				\left(
					\alpha_{v_{h}}I_{h}
					+\alpha_{v_{a}} I_{a}
				\right)
				S_{v}
				-
				\left(
					\mu_{v}
					+r_{k} T_{v}
				\right)
				I_{v}
			\right]
			\\
			&+
			\frac{1}{2}
			\left(
				1-1-1+1
			\right)
			S_{v}^{2}
			\left(
				F_{1}^{2}\theta_{v_{h}}^{2}I_{h}^{2}
				+F_{2}^{2}\theta_{v_{a}}^{2}I_{a}^{2}
			\right)
			\\
			&\leq
			\left(
				S_{v} + I_{v}-K
			\right) rT_{v}
			\left(
				1 - \frac{T_{v}}{K}
			\right)
			+
			\left(
				p_{1} \alpha_{h}H
				+ p_{2} \alpha_{a}A
				- \mu_{v}
			\right) I_{v}
			\\
			&-
			r_{k} T_{v }I_{v}
			+
			\left(
				K\alpha_{v_{h}}
				- p_{1}\mu_{h}
			\right)I_{h}
			+
			\left(
				K \alpha_{v_{a}}
				- p_{2} \mu_{a}
			\right)I_{a}
			\\ %
			&-
			\left(
				p_{1} \alpha_{h}
				+ \alpha_{v_{h}}
			\right)
			I_{v} I_{h}
			-
			\left(
				p_{2} \alpha_{a}
				+ \alpha_{v_{a}}
			\right)
			I_{v} I_{a} ~.
	\end{align*}
	Using $p_{1}$ and $p_{2}$ values, we observe that
	\begin{align*}
		\alpha_{v_{h}}K - p_{1}\mu_{h} &= 0\\
		\alpha_{v_{a}}K - p_{2}\mu_{a} &= 0 ~.
	\end{align*}
	Thus, we obtain
	\begin{align*}
		\mathcal{L}V(I_{h},I_{a},S_{v},I_{v}) 
			&\leq
				\left(
					S_{v} + I_{v} - K
				\right)
				rT_{v}
				\left(
					1 - \frac{T_{v}}{K}
				\right)
				+
				\mu_{v}
				\left(
					\mathcal{R}-1
				\right)
				I_{v} 
				- r_{k} T_{v} I_{v}
				\\ %
				&-
				\left(
					p_{1} \alpha_{h}
					+\alpha_{v_{h}} 
				\right)
				I_{v}I_{h}
				-
				\left(
					p_{2}
					\alpha_{a}
					+\alpha_{v_{a}} 
				\right)
				I_{v} I_{a}~.
	\end{align*}
	Finally, since $\mathcal{R}<1$, we concluded that 
	$\mathcal{L}V(I_{h},I_{a},S_{v},I_{v})$ is negative-definite on 
	$\mathbf{D}-\{x^{*}\}$. In this way, we have guaranteed the hypotheses of 
	Theorem~\ref{theo.ap.3}.
\end{proof}

\begin{remark}
	\todo{Introduction for $R_0$}
		In mathematical epidemiology context, the basic reproduction number 
	($\mathcal{R}_0$) yields when an epidemic occurs.
	Applying the method of the next generation matrix \cite{VanDenDriessche2002}, 
	we compute $\mathcal{R}_{0}$ for the deterministic system~\eqref{eq.1}
	\begin{align}\label{eq:R_zero}
		\mathcal{R}_{0} &= 
		\sqrt{ 
			\left(
				\frac{z_{h}\theta_{h}H}{\mu_{v}+r}
			\right)
			\left(
				\frac{z_{h}\theta_{v_{h}}K}{\mu_{h}} 
			\right)
			+\left(
				\frac{z_{a}\theta_{v_{a}}K}{\mu_{a}}
			\right)
			\left(
				\frac{z_{a}\theta_{a}A}{\mu_{v}+r}
			\right)
		}~.
	\end{align}
	We interpret this value as the average number of infected individuals 
	indirectly generated by one infected vector during its whole infectious 
	period. Qualitative analysis of the deterministic system \eqref{eq.1}, 
	shows that if $\mathcal{R}_{0}$ is lower than one, then exist two 
	equilibriums --- the trivial and disease-free points --- while if
	 $\mathcal{R}_{0}$ is greater than one, then there is an endemic fixed point.
	\todo{Think an interpretation}
	Note that threshold parameter $\mathcal{R}_0$ always is lower than 
	$\mathcal{R}$, therefore, if $\mathcal{R}$ is less than one, our stochastic 
	system \eqref{eq.2} preserves this threshold behavior.
	\todo{Can we say something about its stability?}
\end{remark}