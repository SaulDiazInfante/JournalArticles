\paragraph{Motivation}
	Now days, vector diseases represents the principal cause of death in 
many sub development countries. A combination of low resources, climate
variations and lack of efficient health services, amplifies its prevalence. So, 
design tools to planing strategies under real conditions is crucial. However
this phenomena depends on many and very intricate variables. 
By this reason, not exist a general model which  describe or predict in a
acceptable way, the essence of this complex system. Instead, a lot of literature
focus on describe a specific characteristic under restrictive or unrealistic 
conditions. We believe that to incorporate uncertainty is an clever option
~--- resume the effect of many variables as "noise" ~---to produce a more 
realistic model.

	

\todo{Find other diseases, structure and use Chagas 
	like example, leismaniasis, peste (sin tratamiento).}
\paragraph{Previous and related work}
	Essentially, in literature exist two main alternatives 
	for incorporate environmental noise. The first alternative, considers as step 
	one, to describe the disease transitions with a discrete Markov chain. Next,
	letting discrete time to zero, one gets a stochastic differential equation 
	(SDE). We refer to [Allen's work] and reference there in. The other approach, 
	perturbs a target parameter $\phi$ of a given  ordinary differential equation 
	(ODE) with a Wienner process. To be precise, for a differential time 
	$(t+dt)$  
	one  stochastically perturbs $\phi dt$ with a Wienner process of intensity 
	$\sigma$. So, one substitute $\phi dt$ by $\phi dt+\sigma dW_t$  to get a SDE
	model. We mention as representative works in this line to [Gray Mao,
	 Greeendhald, Liu Cai].

		Environmental noise could dramatically changes the nature of a deterministic
	equilibrium. For example, Mao et al. (2002) found that 
	even the presence of a tiny noise can suppress a potential population 
	explosion. Hence, it is natural to investigate effects of environment random 
	fluctuations in a population dynamics.

\paragraph{Results}
	We follow ideas of [Schurz] in order to extend a SIS deterministic epidemic 
	model applying a general stochastic perturbation. In this sense our results 
	follows the same style of [Tosun, Cai, Gray. ]. However all mentioned
	literature do not consider multi host structure o vectorial disease 
	transmission. In this line we direct our research. Concretely, 
	our main contributions are:
	\begin{itemize}
		\item 
			The study a disease transmitted by vector with two different hosts, which 
			is a new approach on epidemic SDE models.
		\item
			We consider a family of functions dependent on the states variables
			as noise intensities. In this way, we obtain sufficient criteria to 
			guarantee extinction and persistence of the disease.
		\item 
			According to our numeric simulations, we observe a mean behavior
			which is very similar to our deterministic base dynamics.
			For example, our numeric mean estimations suggest that deterministic 
			free disease and endemics equilibriums remains close in large time from
			its stochastic extension.
	\end{itemize}
	
\paragraph{Outline}