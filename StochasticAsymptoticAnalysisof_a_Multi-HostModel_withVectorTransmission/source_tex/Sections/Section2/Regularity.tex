	For begin with the analysis of SDE ~\eqref{eq.2}, we guarantee the existence 
	and uniqueness of the solutions, and since we study populations, these 
	solution have to be positive. The following Theorem establish this.
\begin{theorem}\label{thm:regularity}
		Under \Cref{ass:regularity} there is an unique solution 
		$\left( I_{h}(t),I_{a}(t),S_{v}(t),I_{v}(t) \right)$ to SDE~\eqref{eq.2} 
		for $t\geq 0$ and the 
	solution will remain in $\mathbf{D}$ with probability 1, namely $\left( 
	I_{h}(t),I_{a}(t),S_{v}(t),I_{v}(t) \right)\in \mathbf{D}$ for all $t\geq 0$ almost surely.
\end{theorem}
\begin{proof}
	Let us first outline of the proof. We first going to guarantee the local 
existence and uniqueness of the solution, after that, we prove the globality 
of this and the invariance of $\mathbf{D}$ employing  the Corollary 3.1 of 
\citet[p.~76]{Khasminskii2012}.

	Let 
	$
		X(t)=( I_{h}(t),I_{a}(t),S_{v}(t),I_{v}(t) )
	$,
$
	F_{1} = F_{1}\left(I_{h}(t),I_{a}(t),S_{v}(t),I_{v}(t)\right)
$,
$
	F_{2} = F_{2}\left(I_{h}(t),I_{a}(t),S_{v}(t),I_{v}(t)\right)
$.
For each $n\in \mathbb{N}$ define,
\begin{align*}
	\mathbf{D}_{n}
		:= 
		\{ 
			\left( 
				I_{h},I_{a},S_{v},I_{v} 
			\right)\in \mathbb{R}^{4}:
			\quad
			& 
			e^{-n} < I_{h} < H-e^{-n},
			\quad
			e^{-n} < I_{a} < A-e^{-n}, \\
			&
			e^{-n} < S_{v} < K-e^{-n},
			e^{-n} < I_{v} < K-e^{-n},
			S_{v}+I_{v} \leq K 
		 \}.
\end{align*}

Let us denote by $\tau(\mathbf{D}_{n})$  the random time of first 
exit of stochastic process $\left( I_{h}(t),I_{a}(t),S_{v}(t),I_{v}(t) \right)$ from the 
set $\mathbf{D}_{n}$. 
Since $b_{0}(X(t))$ and $b_{1}(X(t))$ are locally Lipschitz-continuous and satisfy 
linear growth condition on $\mathbf{D}_{n}$, there is an unique local solution on 
$t\in [0,\tau(\mathbf{D}_{n}))$ for any initial value 
$\left( I_{h}(0),I_{a}(0),S_{v}(0),I_{v}(0) \right)\in \mathbf{D}_{n}$.\\

	To prove the global existence of the solution, we follow ideas of 
\cite{Schurz2015}. 
Let,
\begin{align*}
	V\left( I_{h},I_{a},S_{v},I_{v} \right) 
		&= I_{h} 
		+ (H-I_{h}) 
		- \ln(H-I_{h}) 
		+ I_{a} + (A-I_{a}) -
		\ln(A-I_{a}) + (K-S_{v})\\
		&\quad -\ln (K-S_{v}) 
		+ S_{v} 
		- \ln S_{v} 
		+ I_{v} 
		+ (K-I_{v}) 
		- \ln(K-I_{v}),
\end{align*}

defined on
\begin{align*}
	\widetilde{\mathbf{D}} 
		:=\{ 
			\left( 
				I_{h}, I_{a}, S_{v},I_{v} 
			\right) \in \mathbb{R}^{4};  
			\ 0\leq t: \ 
			0<I_{h}< H,
			\ 
			0<I_{a}< A,
			\ 
			0< S_{v}< K,
			0< I_{v}< K,
			\ 
			S_{v}+I_{v}\leq K
		\}.
\end{align*}

Since $y-\ln y \geq 1$ for all $y\in \left(0,\infty \right)$, it follows that 
$
	V \left( 
		I_{h},I_{a},S_{v},I_{v} 
	\right) \geq 5
$
for
$
	\left(
		I_{h},I_{a},S_{v},I_{v} 
	\right)\in \widetilde{\mathbf{D}}
$
Also, we define
\begin{multline*}
	c = \frac{1}{5}
		\left( 
			\alpha_{h}
			+\alpha_{a}
		\right) K 
		+ 
		\frac{2}{5}
		\left(
			\alpha_{v_{h}}H
			+\alpha_{v_{a}}A
		\right)
		+\frac{1}{5}
		\left( 
			2\Lambda_{v}+r
		\right) 
		+
		\frac{1}{10}
		\sup
		\limits_{(I_{h},I_{a},S_{v},I_{v})
		\in 
		\widetilde{\mathbf{D}}}  
		\left\{
			\left(
				\theta_{h}^{2}K^{4}
				+3 \theta_{v_{h}}^{2}H^{2}K^{2} 
			\right)G_{1}^{2}
		\right\}\\
		+
		\frac{1}{10}
		\sup
		\limits_{(I_{h},I_{a},S_{v},I_{v})
		\in 
		\widetilde{\mathbf{D}}} 
		\left\{
			\left(
				\theta_{a}^{2}K^{4}
				+3\theta_{v_{a}}^{2}A^{2} K^{2} 
			\right)G_{2}^{2}
		\right\} ~.
\end{multline*}

	Applying the infinitesimal generator (eq.~\eqref{eq.ap.2}) to our
Liapunov function 
$
	V
	\left( 
		I_{h},I_{a},S_{v},I_{v} 
	\right)
$, yields
\begin{align*}
	\mathcal{L}V\left( I_{h},I_{a},S_{v},I_{v} \right) &= 
		\left[
			\alpha_{h}I_{v}\left(H-I_{h}\right)
			-\mu_{h}I_{h}\right] \frac{\partial V}{\partial I_{h}} 
			+ 
			\left[
				\alpha_{a}I_{v}
					\left(
						A-I_{a}
					\right)
					-\mu_{a}I_{a}
			\right]
			\frac{\partial V}{\partial I_{a}}
		\\%
		&+ 
		\left[
			\Lambda_{v}T_{v}
			-
			\left(
				\alpha_{v_{h}}I_{h}
				+\alpha_{v_{a}}	I_{a}
			\right) S_{v}
			-
			\left(
				\mu_{v}+r_{k}
				T_{v}
			\right) S_{v}
		\right]
		\frac{\partial V}{\partial S_{v}}
		\\ %
		&+ 
		\left[
			\left(
				\alpha_{v_{h}}I_{h}+\alpha_{v_{a}}I_{a}
			\right) S_{v}
			-
			\left(
				\mu_{v}+r_{k}T_{v}
			\right) I_{v}
		\right] 
		\frac{\partial V}{\partial I_{v}}
		\\ %
		&+
		\frac{1}{2}
		F_{1}^{2}
		\theta_{h}^{2}I_{v}^{2}
		\left(
			H-I_{h}
		\right)^{2}
		\frac{\partial^{2}V}{\partial I_{h}^{2}}
		+\frac{1}{2}
		F_{2}^{2}
		\theta_{a}^{2}
		I_{v}^{2}
		\left(
			A-I_{a}
		\right)^{2}
		\frac{\partial^{2}V}{\partial I_{a}^{2}}
		\\ %
		&+
		\frac{1}{2}
		\left[
			F_{1}^{2}
			\theta_{v_{h}}^{2}I_{h}^{2}S_{v}^{2} 
			+F_{2}^{2}\theta_{v_{a}}^{2}I_{a}^{2}S_{v}^{2}
		\right]
		\left(
			\frac{\partial^{2}V}{\partial S_{v}^{2}}
			+\frac{\partial^{2}V}{\partial I_{v}^{2}}
		\right)\\[0.25cm]
		&=
		\left(
			\frac{\alpha_{h}I_{v}
				\left(
					H-I_{h}
				\right)
				-\mu_{h}I_{h}}{H-I_{h}}
		\right) 
		+
		\left(
			\frac{
					\alpha_{a}I_{v}
					\left(
						A-I_{a}
					\right)
					-\mu_{a}I_{a}
			}{A-I_{a}}
		\right)
		\\ %
		&+ 
		\left(
			\frac{
				\Lambda_{v} 
				T_{v}
				-
					\left(
						\alpha_{v_{h}}I_{h}
						+\alpha_{v_{a}}I_{a}
					\right)
					S_{v}
					-
					\left(
						\mu_{v}+r_{k}T_{v}
					\right)S_{v}
				}{-S_{v}}
		\right)
		\\ %
		&+ 
		\left(
			\frac{
				\Lambda_{v} 
				T_{v}
				-
				\left(
					\alpha_{v_{h}}I_{h}
					+\alpha_{v_{a}}I_{a}
				\right)S_{v}
				-
				\left(
					\mu_{v}
					+r_{k}T_{v}
				\right)
				S_{v}
			}{K-S_{v}}
		\right)
		\\ %
		&+ 
		\left(
			\frac{
				\left(
					\alpha_{v_{h}}I_{h}
					+\alpha_{v_{a}}I_{a}
				\right)S_{v}
				-
				\left(
					\mu_{v}+r_{k}T_{v}
				\right) I_{v}
			}{K-I_{v}}
		\right)\\
		&+ 
		\frac{1}{2}
		\left[
			\left(
				\frac{
					F_{1}^{2}\theta_{h}^{2}I_{v}^{2}
					\left(
						H-I_{h}
					\right)^{2}
				}{(H-I_{h})^{2}}
			\right)
			+
			\left(
				\frac{
					F_{2}^{2} \theta_{a}^{2}I_{v}^{2}
					\left(
						A-I_{a}
					\right)^{2}
				}{(A-I_{a})^{2}}
			\right)
		\right]
		\\ %
		&+
		\frac{1}{2}
		\left[
			F_{1}^{2}\theta_{v_{h}}^{2}I_{h}^{2}S_{v}^{2} 
			+F_{2}^{2}\theta_{v_{a}}^{2}I_{a}^{2}S_{v}^{2}
		\right]
		\left(
			\frac{1}{S_{v}^{2}} 
			+ \frac{1}{(K-S_{v})^{2}} 
			+ \frac{1}{(K-I_{v})^{2}}
		\right) ~.
\end{align*}
Substituting $F_{i}=I_{v}G_{i}$ on the above relation, we obtain 
\begin{align*}
	\mathcal{L}V
	\left(
		I_{h},I_{a},S_{v},I_{v}
	\right)
	&=
		\left(
			\frac{
				\alpha_{h}I_{v}
				\left(
					H-I_{h}
				\right)
				-\mu_{h}I_{h}
			}
			{H-I_{h}}
		\right) + 
			\left(
				\frac{
					\alpha_{a}I_{v}
					\left(
						A-I_{a}
					\right)
					-\mu_{a}I_{a}
				}
			{A-I_{a}}
		\right)
		\\ %
		&+ 
		\left(
			\frac{
				\Lambda_{v} T_{v}
				-s
				\left(
					\alpha_{v_{h}}I_{h}+\alpha_{v_{a}}I_{a}
				\right)
				S_{v}
				-
				\left(
					\mu_{v}+r_{k}T_{v}
				\right)S_{v}
			}
			{-S_{v}}
		\right)
		\\ %
		&+ 
		\left(
			\frac{
				\Lambda_{v}T_{v}
				-
				\left(
					\alpha_{v_{h}}I_{h}
					+\alpha_{v_{a}}I_{a}
				\right)
				S_{v}
				-
				\left(
					\mu_{v}+r_{k}T_{v}
				\right)
				S_{v}
			}{K-S_{v}}
		\right)
		\\ %
		&+ 
		\left(
			\frac{
				\left(
					\alpha_{v_{h}}I_{h}
					+\alpha_{v_{a}}I_{a}
				\right)
				S_{v}
				-
				\left(
					\mu_{v}
					+r_{k}T_{v}
				\right)
				I_{v}
			}
			{K-I_{v}}
		\right)
		\\ %
		&+
		\frac{1}{2}
		\left[
			\left(
				\frac{
					I_{v}^{2}G_{1}^{2} \theta_{h}^{2}I_{v}^{2}
					\left(
						H-I_{h}s
					\right)^{2}
				}
				{(H-I_{h})^{2}}
			\right)
			+
			\left(
				\frac{
					I_{v}^{2}G_{2}^{2} \theta_{a}^{2} I_{v}^{2}
					\left(
						A-I_{a}
					\right)^{2}}
					{(A-I_{a})^{2}}
			\right)
		\right]
		\\ %
		&+
		\frac{1}{2}
		\left[
			I_{v}^{2}G_{1}^{2} \theta_{v_{h}}^{2}I_{h}^{2}S_{v}^{2}
			+I_{v}^{2}G_{2}^{2}\theta_{v_{a}}^{2}I_{a}^{2}S_{v}^{2}
		\right]
		\left(
			\frac{1}{S_{v}^{2}} 
			+\frac{1}{(K-S_{v})^{2}} 
			+\frac{1}{(K-I_{v})^{2}}
		\right) ~.
\end{align*}
Discarding negative terms, we bound the above relation by
\begin{align*}
	\mathcal{L}V
	\left(
		I_{h},I_{a},S_{v},I_{v} 
	\right)
	&\leq
	\left(
		\alpha_{h}
		+\alpha_{a}
	\right)
	I_{v} 
	+ 
	2
	\left(
		\alpha_{v_{h}}I_{h}
		+\alpha_{v_{a}}I_{a}
	\right)
	+
	\left(
		\mu_{v} + r_{k}T_{v}
	\right)
	\\ %
	&+ 
	\left( 
		r+\Lambda_{v}
	\right) 
	+ \frac{1}{2}
	\left(
		G_{1}^{2} \theta_{h}^{2}
		+G_{2}^{2} \theta_{a}^{2}
	\right)
	I_{v}^{4}\\
	&+
	\frac{1}{2}
	\left( 
		2I_{v}^{2}G_{1}^{2} \theta_{v_{h}}^{2}I_{h}^{2}
		+ 2I_{v}^{2}G_{2}^{2}\theta_{v_{a}}^{2}I_{a}^{2}
		+ G_{1}^{2}\theta_{v_{h}}^{2}I_{h}^{2}S_{v}^{2} 
		+ G_{2}^{2}\theta_{v_{a}}^{2}I_{a}^{2}S_{v}^{2}
	\right)
	\\
	&\leq 
		\left( 
			\alpha_{h} + \alpha_{a} 
		\right)
		K + 2
		\left(
			\alpha_{v_{h}}H
			+\alpha_{v_{a}}A
		\right)
		+
		\left( 
			2\Lambda_{v}+r
		\right)
		\\ %
	&+ 
	\sup
	\limits_{
		(I_{h},I_{a},S_{v},I_{v})\in \widetilde{\mathbf{D}}
	} 
	\frac{1}{2} 
	\left\{
		\left( 
			\theta_{h}^{2}K^{4}
			+3\theta_{v_{h}}^{2}H^{2}K^{2} 
		\right)
		G_{1}^{2}
	\right\}\\
	&+ 
	\sup
		\limits_{
			(I_{h},I_{a},S_{v},I_{v})\in \widetilde{\mathbf{D}}
		}
		\frac{1}{2} s
		\left\{
			\left( 
				\theta_{a}^{2}K^{4}
				+3\theta_{v_{a}}^{2}A^{2}K^{2} 
			\right)
			G_{2}^{2}
		\right\}
	\\
	&= 5c~.
\end{align*}
As we know, 
	$
		V\left( I_{h},I_{a},S_{v},I_{v} \right)\geq 5
	$ for all 
	$
		\left( I_{h},I_{a},S_{v},I_{v} \right)\in \widetilde{\mathbf{D}}
	$, and since 
	$
		5c \geq V
		\left( 
			I_{h},I_{a},S_{v},I_{v}
		\right)
	$, 
	we deduce that 
	$
		cV
		\left( 
			I_{h},I_{a},S_{v},I_{v} 
		\right)
		\geq 
		\mathcal{L}V
		\left( 
			I_{h},I_{a},S_{v},I_{v} 
		\right)
	$. 
%

	Now we construct a crescent collection of subsets of $\mathbf{D}$ in order to
satisfies the conditions of \Cref{theo.ap.2}.
For each $n \in \N$ define
$$
	\mathbf{D}_n:=[e^{-n}, H-e^{-n}] \times
			[e^{-n}, A-e^{-n}] \times
			[e^{-n}, K-e^{-n}] \times
			[e^{-n}, K-e^{-n}].
$$
Let 
$$
	\widetilde{\mathbf{D}}\backslash\mathbf{D}_{n} 
	= (0,e^{-n})
	\bigcup (H-e^{-n},H)
	\times (0,e^{-n})
	\bigcup (A-e^{-n},A)
	\times (0,e^{-n})
	\bigcup(K-e^{-n},K)
	\times (0,e^{-n})
	\bigcup (K-e^{-n},K) .
$$ 
Since the real valued function $f(x):= x - \ln(x)$ satisfies
$$
	f(x) \geq 1,\qquad \forall x>0,
$$
we deduce that
\begin{align*}
	V
	\left(
		 I_{h},I_{a},S_{v},I_{v}
	\right)
	&\geq (K-S_{v})
	- \ln(K-S_{v})+ S_{v} 
	- \ln S_{v}+3,%\\
	&\text{ for all }
		\left( 
	I_{h},I_{a},S_{v},I_{v} 
	\right)
	\in 
	\widetilde{\mathbf{D}}
	\backslash\mathbf{D}_{n}.
\end{align*}
Moreover, the function $f(x)$ is increasing when 
$x\in (K-e^{-n},K)$, then
\begin{align*}
	V
	\left( 
		I_{h},I_{a},S_{v},I_{v} 
	\right) 
	&\geq 
		(K-K+e^{-n})- \ln(K-K+e^{-n}) + 4,\\
	&\geq e^{-n} + n + 4,\\
	&> n+4.
\end{align*}
Likewise, note that $f(x)$ decreases when $x\in (0,e^{-n})$, so
\begin{align*}
	V
	\left( 
		I_{h},I_{a},S_{v},I_{v} 
	\right) &\geq e^{-n}- \ln(e^{-n}) + 4,\\
&\geq e^{-n} + n + 4,\\
&> n+4.
\end{align*}
Thus,
\begin{align*}
	V
		\left( 
			I_{h},I_{a},S_{v},I_{v} 
		\right) > n+4,\qquad 
		\text{for all } 
		\left( 
			I_{h},I_{a},S_{v},I_{v} 
		\right)
		\in \widetilde{\mathbf{D}} \backslash\mathbf{D}_{n}~.
\end{align*}
Hence, combining the  above inequalities we arrive at,
\begin{equation}\label{eqn:v_condition}
	\inf_{
			\left(
				I_{h},I_{a},S_{v},I_{v} 
			\right) \in 
			\widetilde{\mathbf{D}}
			\backslash
			\mathbf{D}_{n}}
		V
		\left( 
			I_{h},I_{a},S_{v},I_{v} 
		\right) > n+4, \quad \text{for each } n\in \N ~.
\end{equation}
%
	Next we prove that $( I_{h}(t),I_{a}(t),S_{v}(t),I_{v}(t) )$ remains in 
$\widetilde{D}$. Let 
$
	W
	\left( 
		I_{h},I_{a},S_{v},I_{v},t 
	\right) = 
	e^{-c(t-t_{0})}
	V
	\left(
		I_{h},I_{a},S_{v},I_{v} 
	\right)
$
defined on 
$
	\widetilde{\mathbf{D}}\times [t_{0},\infty)
$ for $t_{0}\geq 0$ 
fixed. Denote by $\tau(\mathbf{D}_n)$ the first exit time of solution
$
\left(
	I_{h},I_{a},S_{v},I_{v} 
\right)
$ from the set $\mathbf{D}_n$
and 
$
	\tau_{n}(t) := \min \{ t,\tau(\mathbf{D}_{n})\}
$. 
Since 
$
	\mathcal{L}V
	\left( 
		I_{h},I_{a},S_{v},I_{v} 
	\right)
	\leq 
	cV
	\left( 
		I_{h},I_{a},S_{v},I_{v} 
	\right)
$, 
we have that, 
$
	\mathcal{L}
	W
	\left( 
		I_{h},I_{a},S_{v},I_{v},t 
	\right)
	\leq 0
$ for 
$
	\left( 
		I_{h},I_{a},S_{v},I_{v},t 
	\right)
	\in 
	\widetilde{\mathbf{D}}
	\times [t_{0},\infty)
$.
In order  to get a upper bound for 
$$
	\mathbb{E}W
	\left( 
		I_{h}(\tau_{n}),I_{a}(\tau_{n}),S_{v}(\tau_{n}),I_{v}(\tau_{n}),\tau_{n}  
	\right),
$$ 
we apply the Dynkin's formula, then
\begin{align*}
	\mathbb{E}
	W
	\left( 
		I_{h}(\tau_{n}),I_{a}(\tau_{n}),S_{v}(\tau_{n}),I_{v}(\tau_{n}),\tau_{n}  
	\right) 
	&= \mathbb{E}
	W
	\left( 
		I_{h}(t_{0}),I_{a}(t_{0}),S_{v}(t_{0}),I_{v}(t_{0}),t_{0}  
	\right)\\
	&+ 
	\mathbb{E}
	\int_{\tau_{n}}^{t_{0}}
		\mathcal{L} W 
		\left(
			I_{h}(s),I_{a}(s),S_{v}(s),I_{v}(s),s
		\right)ds
	\\ %
	&\leq 
	\mathbb{E}
	W
	\left( 
		I_{h}(t_{0}),I_{a}(t_{0}),S_{v}(t_{0}),I_{v}(t_{0}),t_{0}
	\right)
	\\ %
	&= 
	\mathbb{E}
	V
	\left( 
		I_{h}(t_{0}),I_{a}(t_{0}),S_{v}(t_{0}),I_{v}(t_{0})
	\right).
\end{align*}
Since 
$
	\mathbf{D}_{n}
	\subset \mathbf{D}_{n+1} 
$, 
we have that 
\begin{align*}
	\mathbb{P}
	\left(
		\tau(\widetilde{\mathbf{D}}) < t 
	\right)
	&\leq 
	\mathbb{P}
	\left(
		\tau(\widetilde{\mathbf{D}}_{n})<t 
	\right)
	\\ %
	&= 
	\mathbb{E}
	(
		\mathbbm{1}_{\tau_{n}<t})
	\\ %
	&\leq 
	\mathbb{E}
	\left( 
		e^{c(t-\tau_{n})}
		\frac{
			V
			\left(
				I_{h}(\tau_{n}),I_{a}(\tau_{n}),S_{v}(\tau_{n}),I_{v}(\tau_{n})
			\right)
		}{
			\inf\limits_{
				\left( 
					I_{h},I_{a},S_{v},I_{v}
				\right) 
				\in 
				\widetilde{\mathbf{D}}
				\backslash \mathbf{D}_{n}
			}
			V\left( 
				I_{h},I_{a},S_{v},I_{v} 
			\right)} 
		\mathbbm{1}_{\tau_{n}<t}
	\right)\\
	&\leq 
		\frac{e^{c(t-t_{0})}
			\mathbb{E}V
			\left( 
				I_{h}(t_{0}),I_{a}(t_{0}),S_{v}(t_{0}),I_{v}(t_{0})
			\right)
		}
		{
			\inf
			\limits_{
				\left( 
					I_{h},I_{a},S_{v},I_{v} 
				\right)
				\in
				\widetilde{\mathbf{D}}
				\backslash 
				\mathbf{D}_{n}} 
				V
				\left( 
					I_{h},I_{a},S_{v},I_{v} 
				\right)}
		\\ %
		&\leq
		\frac{
			e^{c(t-t_{0})}
			\mathbb{E} V
			\left(
				I_{h}(t_{0}),I_{a}(t_{0}),S_{v}(t_{0}),I_{v}(t_{0})
			\right)}
		{n+4},
\end{align*}
then, letting $n \to \infty$, we see that
$\mathbb{P}(\tau_{n}<t)\rightarrow 0$, 
for all 
$
	\left( 
		I_{h}(t_{0}),I_{a}(t_{0}),S_{v}(t_{0}),I_{v}(t_{0})
	\right)\in \mathbf{D}_{n}
$ 
and fixed $t\in [t_{0},\infty)$.
Furthermore, note that
\begin{align*}
	\P(\tau(\widetilde{\mathbf{D}})<t)
		&=
			\lim
			\limits_{n\rightarrow\infty}
			\mathbb{P}(\tau(\mathbf{D}_{n})<t)=0,
\end{align*}
so, we conclude that,
\begin{align*}
	\mathbb{P}(\tau(\widetilde{\mathbf{D}})=\infty)=1.
\end{align*}
With this, we have proved that for all $t \geq 0$ and initial
condition on $\widetilde{\mathbf{D}}$, the solution
$\left(I_{h},I_{a},S_{v},I_{v}\right)$
remains on $\widetilde{\mathbf{D}}$ with probability one --- in other words,  
the solution process 
$
	\left\{
		\left(
			I_{h}(t),I_{a}(t),S_{v}(t),I_{v}(t)
		\right), t\geq t_0 \geq 0
	\right\}
$
is a.s. regular (or invariant) on $\widetilde{\mathbf{D}}$
in the sense of \citet[Definition 2.1]{Schurz2007}.
Moreover, for each set 
$
	\mathbf{E}_n:=\{(t,x):t\geq t_{0}, 
	x\in\widetilde{\mathbf{D}}_{n}\}
$, the coefficients $b_{0}(X(t))$, $b_{1}(X(t))$ satisfies all conditions of
\Cref{thm:existence}, then exist a unique continuous
solution 
$
	\left\{
		\left(
			I_{h}(t),I_{a}(t),S_{v}(t),I_{v(t)}
		\right)_n, t\geq t_0 \geq 0
	\right\}
$,
which coincides a.s. with $\left(I_{h},I_{a},S_{v},I_{v}\right)$
up to time $\tau_n$, that is
$$
	\P
		\left\{
			\sup_{t_0\leq t \leq \tau_n}
			\norm{
				\left(
					I_{h}(t),I_{a}(t),S_{v}(t),I_{v}(t)
				\right)
				-
				\left(
					I_{h}(t),I_{a}(t),S_{v}(t),I_{v}(t)
				\right)_n
			}>0
		\right\}=0.
$$
Furthermore, $\widetilde{\mathbf{D}}_{n}$ satisfies conditions of 
\Cref{theo.ap.2}, then we concluded that
$
	\left(
		I_{h}(t),I_{a}(t),S_{v}(t),I_{v(t)}
	\right)
$ is the unique global continuous solution of SDE's \eqref{eq.2} and this
solution is invariant respect to $\widetilde{\mathbf{D}}$.

	Note that sets with constraints as, $I_{h}=0$, $I_{a}=0$, $I_{v}=0$ and 
$S_{v}=K$ are not in $\widetilde{\mathbf{D}}$. So, we discuss these in the 
following cases
\begin{enumerate}[{(CASE-}I)]
\item 
	To fix ideas, we first examine the $I_{h}=0$ constraint.
	In this case SDE~\eqref{eq.2} becomes
	\begin{equation}\label{eq.3}
		\begin{aligned}
			dI_{a} &= 
			\left[
				\alpha_{a}I_{v}\left(A-I_{a}\right)-\mu_{a}I_{a}
			\right]dt 
			+ F_{2}\theta_{a}I_{v}
			\left(
				A-I_{a}
			\right)dW_{2}(t)
			\\
			dS_{v} 
				&= 
					\left[
						\Lambda_{v} T_{v}
						-\alpha_{v_{a}}I_{a}S_{v}
						-
						\left(
							\mu_{v} +r_{k}T_{v}
						\right)S_{v}
					\right]dt 
					-F_{2}\theta_{v_{a}}I_{a}S_{v}dW_{2}(t)
					\\
			dI_{v} 
				&= 
				\left[
					\alpha_{v_{a}}I_{a}S_{v}-\left(\mu_{v} 
					+r_{k}T_{v}\right)I_{v}
				\right]dt 
				+ F_{2}\theta_{v_{a}}I_{a}S_{v}dW_{2}(t) ~.
		\end{aligned}
	\end{equation}
	Let 
	$
		\mathbf{D}_{1} = 
			\{ 
				\left( 
					I_{a},S_{v},I_{v} \right)
					\in \mathbb{R}^{3};
					\ 0\leq t : \ 0< I_{a} < A,\ 0< S_{v} < K,\ 0< I_{v}< K,\ 
					S_{v}+I_{v}\leq K 
				 \}
		$ 
		the set over which is defined the SDE~\eqref{eq.3}, 
		and 
		$$
			V_{1}
			\left( 
				I_{a},S_{v},I_{v} 
			\right) 
			= I_{a} 
				+ (A-I_{a}) 
				- \ln(A-I_{a}) 
				+ (K-S_{v}) 
				-\ln(K-S_{v}) 
				+ S_{v} 
				- \ln S_{v} 
				+ I_{v} 
				+ (K-I_{v}) 
				- \ln(K-I_{v}).
		$$
		Following the above argument, we can prove the 
		invariance of $\mathbf{D}_{1}$, global existence, continuous 
		and uniqueness of the solution. Furthermore, this reasoning applies
		to the cases with $I_{a}=0$ and $I_{v}=0$ constraints.
		\item
			Now, if $S_{v}=K$, then SDE~\eqref{eq.2} becomes
			\begin{equation}\label{eq.7}
				\begin{aligned}
					dI_{h} &= -\mu_{h}I_{h}dt\\
					dI_{a} &= -\mu_{a}I_{a}dt,
				\end{aligned}
			\end{equation}
		since $S_{v}=K$ implies $I_v=0$. Also, note that 
		\Cref{eq.7} is lineal, and consequently has a unique continuous global 
		solution in 
		$
		\mathbf{D}_{5} = \{ \left( I_{h},I_{a} \right)\in \mathbb{R}^{2};\ 
		0\leq t 
		: \ 0< I_{h} < H,\ 0< I_{a} < A\}
		$.
	\end{enumerate}
	The same proof works for sub-domains with combinations of mentioned 
	constraints.
\end{proof}
